% Based on the resources listed below.
% Joseph Van Boxtel, Daniel Brown, Jakob Miner
%
%Copyright 2014 Jean-Philippe Eisenbarth
%This program is free software: you can
%redistribute it and/or modify it under the terms of the GNU General Public
%License as published by the Free Software Foundation, either version 3 of the
%License, or (at your option) any later version.
%This program is distributed in the hope that it will be useful,but WITHOUT ANY
%WARRANTY; without even the implied warranty of MERCHANTABILITY or FITNESS FOR A
%PARTICULAR PURPOSE. See the GNU General Public License for more details.
%You should have received a copy of the GNU General Public License along with
%this program.  If not, see <http://www.gnu.org/licenses/>.

%Based on the code of Yiannis Lazarides
%http://tex.stackexchange.com/questions/42602/software-requirements-specification-with-latex
%http://tex.stackexchange.com/users/963/yiannis-lazarides
%Also based on the template of Karl E. Wiegers
%http://www.se.rit.edu/~emad/teaching/slides/srs_template_sep14.pdf
%http://karlwiegers.com
\documentclass{scrreprt}
\usepackage{listings}
\usepackage{underscore}
\usepackage[bookmarks=true]{hyperref}
\usepackage[utf8]{inputenc}
\usepackage[english]{babel}
\hypersetup{
    bookmarks=false,    % show bookmarks bar?
    pdftitle={Software Requirement Specification},    % title
    pdfauthor={Jean-Philippe Eisenbarth},                     % author
    pdfsubject={TeX and LaTeX},                        % subject of the document
    pdfkeywords={TeX, LaTeX, graphics, images}, % list of keywords
    colorlinks=true,       % false: boxed links; true: colored links
    linkcolor=blue,       % color of internal links
    citecolor=black,       % color of links to bibliography
    filecolor=black,        % color of file links
    urlcolor=purple,        % color of external links
    linktoc=page            % only page is linked
}%
\def\myversion{1.0 }
\date{}
%\title
\usepackage{hyperref}
\begin{document}

\begin{flushright}
    \rule{16cm}{5pt}\vskip1cm
    \begin{bfseries}
        \Huge{SOFTWARE REQUIREMENTS\\ SPECIFICATION}\\
        \vspace{1.9cm}
        for\\
        \vspace{1.9cm}
       CS320 Project\\
        \vspace{1.9cm}
        \LARGE{Version \myversion approved}\\
        \vspace{1.9cm}
        Prepared by:\\
        Team Name: Our Team!
        \begin{center}
            \begin{tabular}{|c|c|c|}
                \hline
        	    Name & SID & Email\\
                \hline
        	    Joseph Van Boxtel & SID & joseph.vanboxtel@wsu.edu\\
                \hline
        	    Jakob Miner & SID & jakob.miner@wsu.edu\\
                \hline
            \end{tabular}
        \end{center}

        \vspace{1.9cm}
        \today\\
    \end{bfseries}
\end{flushright}

\tableofcontents


\chapter*{Revision History}

\begin{center}
    \begin{tabular}{|c|c|c|c|}
        \hline
	    Name & Date & Reason For Changes & Version\\
        \hline
	    Joseph, Jakob & 10/16/19 & Title Page & 0.01\\
        \hline
    \end{tabular}
\end{center}

\chapter{Introduction}

\section{Purpose}

The purpose of this System Requirements Specification documentation is to
provide a complete and thorough definition of the Group Calendar web application.
This includes the purpose of the application and its usage and features. It will
also detail the constraints and limits of the application, and who this
application's intended audience is.


\section{Project Scope}

The software is a web application that allows users to merge and share calendars.
It aims to help people with lots of calendars easily give others access to an
aggregated view. The software will also allow groups of users to plan a meeting
at a time that is available in their calendars. The software won’t determine the
best meeting time, it will simply overlay calendars and/or show shared availability.


\section{Intended Audience and Reading Suggestions}

This document in intended for our professor, developers on the project, and
future customers interested in the product (Not end users). Developers should
read sequentially with a focus on the requirements in chapters three and four.
The professor should focus on chapter two and the use case diagrams. Interested
customers should also focus on chapter two and dig in to the later chapters for
more detail on the specific requirements that this product meets.


\section{Definitions Acronyms and Abbreviations}
$<$Describe the different types of reader that the document is intended for,
such as developers, project managers, marketing staff, users, testers, and
documentation writers. Describe what the rest of this SRS contains and how it is
organized. Suggest a sequence for reading the document, beginning with the
overview sections and proceeding through the sections that are most pertinent to
each reader type.$>$

\section{Document Conventions}
$<$Describe any standards or typographical conventions that were followed when
writing this SRS, such as fonts or highlighting that have special significance.
For example, state whether priorities  for higher-level requirements are assumed
to be inherited by detailed requirements, or whether every requirement statement
is to have its own priority.$>$


\section{References}
$<$List any other documents or Web addresses to which this SRS refers. These may
include user interface style guides, contracts, standards, system requirements
specifications, use case documents, or a vision and scope document. Provide
enough information so that the reader could access a copy of each reference,
including title, author, version number, date, and source or location.$>$


\chapter{Overall Description}

\section{Product Perspective}
The Group Calendar web application is a stand-alone web app. It consists of a
web interface allowing users to see commonly unscheduled time blocks between
users in groups, and a database which stores the information. The database is
broken into 2 categories:
\\1. users
\\  (individual calendars, preferences, groups)
\\2. groups
\\  (members, consolidated calendar, group events)

\section{Product Functionality}
\begin{itemize}
\item Register / log-in functionality - Allow user to create a profile, or log
into an existing profile, where user data will be stored.
\item Import calendar files - Users import existing .ical files, which are
stored in database for creation of group calendars.
\item Manage group membership - Users can view, create, or leave groups. Group
admins may invite users to existing groups.
\item View group-wide common availability - Group members are presented with a
common calendar, showing time blocks where none of the selected users have
events scheduled.
\item Filter users from consideration - Users may narrow the selection of users
being considered when creating and displaying the group availability calendar.
\item Select time scale - User can change the calendar view to differing time 
scales, for example monthly or weekly view.
\item Select group view - Users that are members of multiple groups can select
which group calendar they wish to see.
\end{itemize}

\section{Users and Characteristics}
\begin{itemize}
    \item Registered User – This user has signed up for an account. They have access to
all the basic functionality of creating and managing calendar groupings.
    \item Group Member – Is registered. Accepted an invite to a group.
    \item Group Leader – Is registered. Created a group. Can manage group members.
    \item Guest User – Not a registered user. Was given an iCal link but doesn’t have edit
access. Only access to the iCal feed link.
    \item Admin – Must have special permission and/or access to the database. Performs
maintenance on the database.
\end{itemize}
 \\
Registered Users are more important than guest users and admins.

\section{Operating Environment}
The operating environment for this application is an in-browser application. The
application is intended to run on a desktop web browser, so mobile usability is
not a priority. Due to the in-browser usage of the application, operating system
and platform should not affect usability.

\section{Design and Implementation Constraints}
$<$Describe any items or issues that will limit the options available to the
developers. These might include: corporate or regulatory policies; hardware
limitations (timing requirements, memory requirements); interfaces to other
applications; specific technologies, tools, and databases to be used; parallel
operations; language requirements; communications protocols; security
considerations; design conventions or programming standards (for example, if the
customer’s organization will be responsible for maintaining the delivered
software).$>$

\section{User Documentation}
For the purpose of this application, a complete user manual is unnecessary.
Given the limited number of actions available to each user and simplicity of the
interface,the application should be able to be used by the user immediately,
provided the user is familiar with the scope of the application. For this
reason, a landing page with a brief description of the application, and an FAQ
page will be the extent of the user manual.

\section{Assumptions and Dependencies}
$<$List any assumed factors (as opposed to known facts) that could affect the
requirements stated in the SRS. These could include third-party or commercial
components that you plan to use, issues around the development or operating
environment, or constraints. The project could be affected if these assumptions
are incorrect, are not shared, or change. Also identify any dependencies the
project has on external factors, such as software components that you intend to
reuse from another project, unless they are already documented elsewhere (for
example, in the vision and scope document or the project plan).$>$


\chapter{Specific Requirements}

\section{External Interface Requirements}
For the primary user interface, the user will be presented with a calendar view.
The calendar can be presented in either a monthly or weekly view. Along the left
edge of the calendar will be clickable tabs representing the different groups
the user is a member of, which by clicking, the user can change which group
focus they are interested in viewing. Along the right side will be similar tabs
representing the various users and sub-groups that are members of the same
group that is selected. These tabs can be selected to filter out selected users
from consideration. At the top right, the user can find links to manage their
settings and profile.
\section{Functional Requirements}

\section{Behavior Requirements}


\chapter{Nonfunctional Requirements}

\section{Performance Requirements}
$<$If there are performance requirements for the product under various
circumstances, state them here and explain their rationale, to help the
developers understand the intent and make suitable design choices. Specify the
timing relationships for real time systems. Make such requirements as specific
as possible. You may need to state performance requirements for individual
functional requirements or features.$>$

\section{Safety and Security Requirements}
This application, by design, is safe and secure to use. By importing .ical
copies of the user's personal calendars, the application cannot change or affect
the user's master calendar in any way. Additionally, the actual appointment
titles and information is never accessed, only the date and times, so any
sensitive or personal information is not visible to others using the application.

\section{Software Quality Attributes}
$<$Specify any additional quality characteristics for the product that will be
important to either the customers or the developers. Some to consider are:
adaptability, availability, correctness, flexibility, interoperability,
maintainability, portability, reliability, reusability, robustness, testability,
and usability. Write these to be specific, quantitative, and verifiable when
possible. At the least, clarify the relative preferences for various attributes,
such as ease of use over ease of learning.$>$

\chapter{Other Requirements}
$<$Define any other requirements not covered elsewhere in the SRS. This might
include database requirements, internationalization requirements, legal
requirements, reuse objectives for the project, and so on. Add any new sections
that are pertinent to the project.$>$

\end{document}
